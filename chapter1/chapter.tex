\ifx\allfiles\undefined
\documentclass[12pt, a4paper, oneside, UTF8]{ctexbook}
\def\path{../config}
\usepackage{amsmath}
\usepackage{amsthm}
\usepackage{amssymb}
\usepackage{graphicx}
\usepackage{mathrsfs}
\usepackage{mathtools}
\usepackage{pifont}  % 多种符号
\usepackage{esint}  % 各种积分号
\usepackage{upgreek}  % 正体希腊字母
\usepackage{enumitem}
\usepackage{geometry}
\usepackage[colorlinks, linkcolor=black]{hyperref}
\usepackage{stackengine}
\usepackage{yhmath}
\usepackage{extarrows}
\usepackage{unicode-math}
\usepackage{caption}
\usepackage{ulem}  % 各种标记线
\usepackage{xargs}  % 多缺省命令
\usepackage{bm}  %  加粗数学字符 

\usepackage{fancyhdr}
\usepackage[dvipsnames, svgnames]{xcolor}
\usepackage{listings}
\usepackage{zhlipsum}  % 随机中文文本, 测试用
\usepackage{lipsum}  % 随机英文文本, 测试用
\usepackage{blindtext}  % 随机科技论文,测试用
\definecolor{mygreen}{rgb}{0,0.6,0}
\definecolor{mygray}{rgb}{0.5,0.5,0.5}
\definecolor{mymauve}{rgb}{0.58,0,0.82}

\graphicspath{{figure/}, {../figure/}, {config/}, {../config/}}

% 设置图标签格式
\captionsetup[figure]{
    labelfont={bf},labelformat={default},labelsep=period,name={Fig.}}

\linespread{1.6}

\geometry{
    top=25.4mm, 
    bottom=25.4mm, 
    left=20mm, 
    right=20mm, 
    headheight=2.17cm, 
    headsep=4mm, 
    footskip=12mm
}

\setenumerate[1]{itemsep=5pt, partopsep=0pt, parsep=\parskip, topsep=5pt}
\setitemize[1]{itemsep=5pt, partopsep=0pt, parsep=\parskip, topsep=5pt}
\setdescription{itemsep=5pt, partopsep=0pt, parsep=\parskip, topsep=5pt}

\lstset{
    language=Mathematica,
    basicstyle=\tt,
    breaklines=true,
    keywordstyle=\bfseries\color{NavyBlue}, 
    emphstyle=\bfseries\color{Rhodamine},
    commentstyle=\itshape\color{black!50!white}, 
    stringstyle=\bfseries\color{PineGreen!90!black},
    columns=flexible,
    numbers=left,
    numberstyle=\footnotesize,
    frame=tb,
    breakatwhitespace=false,
} 

\numberwithin{equation}{section}  % 公式按节编号
\usepackage{tcolorbox}

\tcbuselibrary{most}

% 定义单独编号,其他四个共用一个编号计数 (原版),这里只列举了五种,其他可类似定义(未定义的使用原来的也可)
\newtcbtheorem[number within=section]{defn}
    {定义}{colback=Salmon!20, colframe=Salmon!90!Black, fonttitle=\bfseries}{def}

\newtcbtheorem[number within=section]{lemma}
    {引理}{colback=OliveGreen!10, colframe=Green!70, fonttitle=\bfseries}{lem}

% 使用另一个计数器可加参数 use counter from=lemma
\newtcbtheorem[number within=section]{them}
    {定理}{colback=SeaGreen!10!CornflowerBlue!10, 
           colframe=RoyalPurple!55!Aquamarine!100!,
           fonttitle=\bfseries}{them}

\newtcbtheorem[number within=section]{criterion}
    {准则}{colback=green!5, colframe=green!35!black, fonttitle=\bfseries}{cri}

\newtcbtheorem[number within=section]{corollary}
    {推论}{colback=Emerald!10, colframe=cyan!40!black, fonttitle=\bfseries}{cor}

\newtcbtheorem[number within=section]{proposition}
    {命题}{colback=red!5,colframe=red!75!black,fonttitle=\bfseries}{prop}
% red!5,colframe=red!75!black 警告框
% 使用格式是\begin{***}{}{} \end{***} ,需要两个 {}{} ,可以不填,但要有.
% 第一个 {} 填入别名 第二个为引用的 label 
% 引用方法为 \ref{def:xxx}

\newtheorem{example}{\indent \color{SeaGreen}{例}}[section]
\theoremstyle{plain}
\newtheorem*{rmk}{\indent 注}
\renewenvironment{proof}{\indent\textcolor{SkyBlue}{\textbf{证明:}}\;}{\qed\par}
\newenvironment{solution}{\indent\textcolor{SkyBlue}{\textbf{解:}}\;}{\qed\par}
% \def\d{\mathrm{d}}
\def\Rs{\mathbb{R}}
\def\Cs{\mathbb{C}}
\def\Zs{\mathbb{Z}}
\def\Ns{\mathbb{N}}
\newcommand\mpi{\uppi}
\newcommand\mi{\mathrm{i}} 
\newcommand\me{\mathrm{e}}
\newcommand*{\dif}{\mathop{}\!\mathrm{d}}  % 微分算符 d
% \newcommand{\bs}[1]{\boldsymbol{#1}}
\newcommand{\ora}[1]{\overrightarrow{#1}}
\newcommand{\myspace}[1]{\par\vspace{#1\baselineskip}}
\newcommand{\xrowht}[2][0]{\addstackgap[.5\dimexpr#2\relax]{\vphantom{#1}}}
\newenvironment{ca}[1][1]{\linespread{#1} \selectfont \begin{cases}}{\end{cases}}
\newenvironment{vx}[1][1]{\linespread{#1} \selectfont \begin{vmatrix}}{\end{vmatrix}}
\newcommand{\tabincell}[2]{\begin{tabular}{@{}#1@{}}#2\end{tabular}}
\newcommand{\pll}{\kern 0.56em/\kern -0.8em /\kern 0.56em}
\newcommand{\bit}[1]{\symbfit{#1}}  % 加粗斜体
\newcommand\sech{\mathrm{sech}\,}  % 正割双曲函数
\newcommand\cosech{\mathrm{cosech}\,}  % 余割双曲函数
\newcommand{\Arg}[1]{\mathrm{Arg}\;#1}  % 辐角 Arg
\newcommand{\Div}[1]{\mathrm{div}\;#1}  % 散度 div
\newcommand{\grad}[1]{\mathrm{grad}\;#1}  % 梯度 grad
\newcommand{\Rot}[1]{\mathrm{rot}\;#1}  % 旋度 rot
\newcommand{\re}[1]{\mathrm{Re}\;#1}  % 实部 Re
\newcommand{\im}[1]{\mathrm{Im}\;#1}  % 虚部 Im
\newcommand*{\colorstar}[1][black]{\textcolor{#1}{\ast\quad}}  % 星星开头的重点行 (可改变星星颜色)
% 彩色的文字及下划线
\newcommandx*{\coloruline}[3][1=black, 2=black, usedefault]{
    \textcolor{#1}{\!\uline{\textcolor{#2}{#3}}}}  % 未完成此功能,下划线换行问题

\def\myIndex{0}  % 封面
% \input{\path/cover_package_\myIndex.tex}

\def\myTitle{复变函数}
\def\myAuthor{高峰}
\def\myDateCover{2023 年 6 月 13 日}
\def\myDateForeword{\today}
\def\myForeword{前言}
\def\myForewordText{
    
}
\def\mySubheading{}


\begin{document}
% \input{\path/cover_text_\myIndex.tex}

\newpage
\thispagestyle{empty}
\begin{center}
    \Huge\textbf{\myForeword}
\end{center}
\myForewordText
\begin{flushright}
    \begin{tabular}{c}
        \myDateForeword
    \end{tabular}
\end{flushright}

\newpage
\pagestyle{plain}
\setcounter{page}{1}
\pagenumbering{Roman}
\tableofcontents

\newpage
\pagenumbering{arabic}
\setcounter{chapter}{0}  % 从第一章开始计数
\setcounter{page}{1}

\pagestyle{fancy}
\fancyfoot[C]{\thepage}
\renewcommand{\headrulewidth}{0.4pt}
\renewcommand{\footrulewidth}{0pt}








\else
\fi

\chapter{解析函数}

\section{复数及其运算}

\subsection{复数(Complex)的概念}

\[ \sqrt{-1}=\mi,\quad{\mi}^2=-1 \]
\[z = x + \mi y,\quad x,y\in \Rs \]
\textcolor{blue}{实部} $\re{z} = x$,\textcolor{blue}{虚部} $\im{z} = y$.\\
$z\in\Cs$,$z=\re{z}+(\im{z}) \mi$

复共轭 (Complex Conjugation) (C. C.) $\bar{z}$
\[\bar{z} = x - \mi y,\quad \bar{\mi} =-\mi\]
\begin{equation} \label{eq:共轭关系}
    \left\{ 
    \begin{lgathered} 
        x = \frac{1}{2}(z + \bar{z})\\ 
        y = \frac{1}{2\mi}(z - \bar{z}) 
    \end{lgathered}   
    \right.
\end{equation}

\begin{equation*}
    \left\{ 
    \begin{lgathered} 
        z = x + \mi y\\ 
        z = \rho (\cos \varphi + \mi \sin \varphi)\\
        z = \rho \me^{\mi\varphi}
    \end{lgathered}   
    \right.
\end{equation*}

\subsection{复数的表示}

\noindent (1) 复平面 (\hyperref[fig:复平面]{Fig.~\ref{fig:复平面}})
\begin{figure}
    \centering
    \includegraphics{复平面.pdf}
    \caption{\label{fig:复平面} 复平面}
\end{figure}

\noindent (2) 三角函数 (\hyperref[fig:极坐标系下的复平面]{Fig.~\ref{fig:极坐标系下的复平面}})
\begin{equation*}
    \left\{ 
    \begin{lgathered} 
        x = \rho \cos \varphi \\ 
        y = \rho \sin \varphi, 
    \end{lgathered}   
    \right. \quad
    \left\{ 
    \begin{lgathered} 
        \rho^2 = x^2 + y^2 \\ 
        \tan \varphi = \frac{y}{x}\quad (x\neq 0)
    \end{lgathered}   
    \right.
\end{equation*}
$\rho$ 和 $\varphi$ 满足:
\begin{equation*}
    \left\{ 
    \begin{lgathered} 
        0 \leqslant \rho < \infty\\ 
        -\infty < \varphi < +\infty. 
    \end{lgathered}   
    \right.
\end{equation*}

\begin{figure}
    \centering
    \includegraphics{极坐标系下的复平面.pdf}
    \caption{\label{fig:极坐标系下的复平面} 极坐标系下的复平面}
\end{figure}

\noindent (3) 负指数
\[ z = \rho(\cos\varphi+\mi\sin\varphi)=\rho\me^{\mi \varphi}\]
$\rho=\left\lvert z \right\rvert $,辐角主值 $\arg{z}$ 
满足 $0 \leqslant \arg{z} < 2\mpi$,也有书中为 $-\mpi < \arg{z} \leqslant \mpi$,
\[\varphi=\Arg{z}=\arg{z}+2k\mpi,\quad (k\in \Zs).\]
\colorstar 复数$0$的辐角无意义。
\[\bar{z}=\rho\me^{-\mi \varphi}\]
无限远点:模为无限大的复数与复平面的对应点,记为 $\infty$。$\rho$ 无限大,辐角无意义。\\
测地投影,复数球,Riemann 球
\begin{equation*}
    \arg{z}=
    \left\{ 
    \begin{lgathered} 
        \arctan \frac{y}{x},\quad (x>0,\;y>0)\ \mbox{一象限}\\ 
        \arctan \frac{y}{x},\quad (x>0,\;y<0)\ \mbox{四象限}\\
        \arctan \frac{y}{x}+\mpi,\quad (x<0,\;y>0)\ \mbox{二象限}\\
        \arctan \frac{y}{x}-\mpi,\quad (x<0,\;y<0)\ \mbox{三象限}
    \end{lgathered}   
    \right.
\end{equation*}

\subsection{复数的运算}

\noindent $\mi^2=-1$\\
$\mi^{4n}=1,\quad\mi^{4n+1}=\mi,\quad\mi^{4n+2}=-1,\quad\mi^{4n+3}=-\mi,\quad n\in \Zs.$\\
设 $z_1=x_1+\mi y_1$,$z_2=x_2+\mi y_2$,\\
\ding{172} 若 $z_1=z_2$,则 $x_1=x_2$ 且 $y_1=y_2$;\\
\ding{173} 若 $z=x+y\mi=0$,则 $x=0$ 且 $y=0$

\noindent \textbf{四则运算}\\
加法:$z_1+z_2=(x_1+x_2)+\mi (y_1+y_2)$\\
减法:$z_1-z_2=(x_1-x_2)+\mi (y_1-y_2)$\\
乘法:$z_1 \cdot z_2=x_1x_2-y_1y_2+\mi (x_1y_2+x_2y_1)$
\begin{flalign*}
    \mbox{除法:}\frac{z_2}{z_1} & = \frac{x_2+y_2\mi}{x_1+y_1\mi}&\\
        & = \frac{(x_2+y_2\mi)(x_1-y_1\mi)}{x_1^2+y_1^2}&\\
        & = \frac{x_1x_2+y_1y_2+(x_1y_2-x_2y_1)\mi}{x_1^2+y_1^2}&
\end{flalign*}
\begin{align*}
    \left\lvert z_1+z_2\right\rvert & \leqslant \left\lvert z_1\right\rvert + \left\lvert z_2 \right\rvert\\
    \left\lvert z_1-z_2\right\rvert & \geqslant \left\lvert \left\lvert z_1\right\rvert - \left\lvert z_2 \right\rvert \right\rvert
\end{align*}

\noindent 棣莫弗公式:
\[(\cos \varphi+\mi \sin\varphi)^n=\cos (n\varphi)+\mi\sin(n\varphi). \]
指数形式下的乘除法和开方:
\begin{flalign*}
    & \mbox{设\ }z = \rho \me^{\mi\varphi},\ z_1=\rho_1 \me^{\mi\varphi_1},\ z_2=\rho_2 \me^{\mi\varphi_2}&\\
    & z_1\cdot z_2=\rho_1 \rho_2 \me^{\mi(\varphi_1+\varphi_2)}&\\
    & \frac{z_2}{z_1} = \frac{\rho_2}{\rho_1} \me^{\mi(\varphi_2-\varphi_1)}\quad (z_1 \neq 0)&\\
    & \sqrt[n]{z}=z^\frac{1}{n}=\rho^\frac{1}{n} \me^{\mi(\frac{\varphi+2k\mpi}{n}),\quad (k=0,1,2,\ldots,n-1)
    \mbox{\quad n 个不同的复数}}
\end{flalign*}

\noindent \colorstar 形如 $\arg{\frac{z-\mi}{z+\mi}}$ 的几何意义:\\
到 $\mi$ 和到 $-\mi$ 的复向量辐角主值的差 (或者可以认为是夹角)。\\
\colorstar $z^2 \neq \left\lvert z \right\rvert^2$,两个不是一回事。\\
\colorstar[blue] $g(\mi)^{f(\mi)}$,可以先将 $g(\mi)$ 表示为复指数形式。
例如:$(2\mi)^\mi=(2\me^{(\frac{\mpi}{2}+2k\mpi)\mi})^\mi$。\\
\colorstar[blue] $\mbox{三角函数}(f(\mi))$,可以使用 $\frac{z-\bar{z}}{2}$ 方法计算\\
\colorstar[blue] 复杂的三角函数式可以引入复指数处理,例如P6 3.6, 3.7。

\subsection{复数的几何意义---复向量的意义和应用}

\noindent 平面曲线 $f(x,y)=0$ 的复表示方程 $F(z,\bar{z})=0$。\\
将\hyperref[eq:共轭关系]{公式\,(\ref{eq:共轭关系})}\,代入 $f(x,y)=0$,得:
\[f(\frac{1}{2}(z + \bar{z}),\frac{1}{2\mi}(z - \bar{z}))=0,\mbox{\ 设为\ }F(z,\bar{z})=0\]  %TODO F(z)

\begin{figure}
    \centering
    \includegraphics{复数和差.pdf}
    \caption{\label{fig:复数和差} 复数和差}
\end{figure}

\[ \left\lvert z_1+z_2\right\rvert^2 + \left\lvert z_1-z_2\right\rvert^2 = 
    2 \left\lvert z_1\right\rvert^2 + 2 \left\lvert z_2\right\rvert^2 \]

\section{复变函数}

\[w(z)=u(x,y)+\mi v(x,y),
\quad z=x+y\mi \in \Cs,\;x,y\in \Rs,\;u(x,y),v(x,y)\in \Rs\]
$w=f(z),\quad z\in E,\quad f(z)=u(x,y)+\mi v(x,y)$ (\hyperref[fig:复变函数的对应关系]{Fig.~\ref{fig:复变函数的对应关系}})
\begin{figure}
    \centering
    \includegraphics{复变函数的对应关系.pdf}
    \caption{\label{fig:复变函数的对应关系} 复变函数的对应关系}
\end{figure}

\noindent 区域 (\hyperref[fig:区域]{Fig.~\ref{fig:区域}})---点集\\
\ding{172} 点的邻域:$\left\lvert z-z_0\right\rvert<\varepsilon$ 是以 $z_0$ 为中心,
$\varepsilon$ 为半径的 $z_0$ 的邻域/ $\varepsilon$-邻域。\\
\ding{173} 内点:若 $z_0$ 是点集 $E$ 的内点,则总可以找到 $z_0$ 的一个邻域完全属于 $E$。\\
\ding{174} 外点:若 $z_0$ 是点集 $E$ 的外点,则总可以找到 $z_0$ 的一个邻域完全不属于 $E$。\\
\ding{175} 区域 $\sigma$:\romannumeral1. 全由内点构成;\romannumeral2. 连通性。\\
\ding{176} 界点:\romannumeral1. 不是内点;\romannumeral2. 无论怎样的邻域,其内总含有内点。\\
\ding{177} 边界\,(线) $l$:界点的集合。\\
\ding{178} 闭区域 $\overline{\sigma}=\sigma+l$\\
\ding{179} 单连通区域:区域内任取一条封闭的曲线,其包围的点都是内点。\\
\ding{180} 复连通区域:非单连通区域的区域。(\textcolor{blue}{有洞的}区域)\\
\begin{figure}
    \centering
    \includegraphics{区域.pdf}
    \caption{\label{fig:区域} 区域}
\end{figure}
\colorstar 两种区域内的任两点都可以通过完全在区域内的曲线连接起来。\\
边界的方向:正向---内部在前进方向的左手边。(外边界:逆时针;内边界:顺时针)

\begin{defn}{复变函数的极限}{连续性}
    $w=f(z)$ 在 $z_0$ 点的某邻域有定义。若对 $\forall \varepsilon >0$,总 $\exists \delta >0$,
    当 $0<\left\lvert z-z_0\right\rvert<\delta$ 的区域内,
    使 $\left\lvert f(z)-w_0\right\rvert<\varepsilon$,其中 $w_0$ 为唯一确定的复数,
    则称 $w_0$ 为 $f(z)$ 在 $z_0$ 处的极限。记为 $\lim_{z\rightarrow z_0}f(z)=w_0$。
\end{defn}
\begin{rmk}{}
    $f(z)$ 在 $z_0$ 的任何方向趋于 $w_0$。
\end{rmk}
点连续定义 \rightarrow $\sigma$ 连续定义 \rightarrow $\overline{\sigma}$ 连续定义

\noindent 若 $f(z)$ 在 $\overline{\sigma}$ 上连续,有:\\
(1) $f(z)$ 在 $\overline{\sigma}$ 上有界;\\
(2) $f(z)$ 在 $\overline{\sigma}$ 上一致连续。\\
即对 $\forall \varepsilon >0$,总 $\exists \delta >0$,$\delta$ 无关于 $z$,
对 $\overline{\sigma}$ 中满足 $\left\lvert z_1-z_2\right\rvert<\delta$
的任意两点,使得 $\left\lvert f(z_1)-f(z_2)\right\rvert<\varepsilon$。

\section{导数与解析函数}

\begin{defn}{可微}{}
    若 $f(z)$ 在 $z$ 及其邻域单值连续,
    \[\lim_{\Delta z \to 0} \frac{f(z+\Delta z)-f(z)}{\Delta z}\]
    在 $z$ 处存在且唯一,且与 $\Delta z \to 0$ 的方式无关,则称 $f(z)$ 在 $z$ 处可微 (可导),
    记为 $f'(z)=\frac{\dif f}{\dif z}$。
\end{defn}

\noindent 解析:$f(z)$ 在点 $z_0$ 及其邻域上处处可导。\\
\colorstar[red] 点和邻域都可导 \to 点解析\\
解析函数:若 $f(z)$ 在 $B$ 区域上的每一点都解析,则称 $f(z)$ 是 $B$ 区域上的解析函数。\\
对一个点:可导$\nrightarrow \leftarrow $解析\\
对一个区域 (不包括闭区域):可导$\leftrightharpoons$解析

\begin{them}{柯西黎曼条件 (C-R 条件)}{C-R 条件}
    $f(z)=u+\mi v$ 在 $\sigma$ 区域解析的一个必要不充分条件是:
    \[f'(z)=\lim_{\Delta z \to 0} \frac{\Delta f}{\Delta z}
    =\lim_{\Delta z \to 0} \frac{\Delta u+\mi \Delta v}{\Delta x+\mi \Delta y}.\]
    根据沿实轴方向逼近于 $0$ 和沿平行虚轴方向趋于 $0$ 两种情形,得出:
    \begin{equation*}
        \left\{ 
        \begin{lgathered}
            \frac{\partial u}{\partial x} = \frac{\partial v}{\partial y}\\
            \frac{\partial v}{\partial x} = -\frac{\partial u}{\partial y},
        \end{lgathered}
        \right.
    \end{equation*}
    对区域 $\sigma$ 的每一点都成立,即为 \textcolor{blue}{C-R 条件}。
\end{them}

\colorstar C-R 条件的其他表示:
\begin{equation*}
    \left\{ 
        \begin{lgathered}
            u_x = v_y\\
            u_y = -v_x
        \end{lgathered}
    \right. \mbox{或}
    \left\{ 
        \begin{lgathered}
            u_\rho = \frac{1}{\rho}v_\varphi\\
            u_\varphi = -\rho v_\rho.
        \end{lgathered}
    \right. 
\end{equation*}

\noindent \textbf{求导规则}\\
$(c_1 f_1(z)+c_2 f_2(z))'=c_1 f_1'(z)+c_2 f_2'(z)$\\
$(f_1(z) f_2(z))'=f_1'(z)f_2(z)+f_1(z)f_2'(z)$\\
$(\frac{f_2(z)}{f_1(z)})'=\frac{f_2'(z)f_1(z)-f_1'(z)f_2(z)}{f_1^2(z)}$\\
$(F[G(z)])'=F'[G(z)]G'(z)$

\noindent 在 $\sigma$ 处处可导的一个充要条件:\\
\ding{172} $u_x$,$u_y$,$v_x$,$v_y$ 存在且连续 ($u$,$v$ 在 $\sigma$ 上存在连续的 $1$ 阶偏导数);\\
\ding{173} C-R 条件 (在 $\sigma$ 上)。

\noindent 若 $f$ 在 $\sigma$ 上处处可导,有:
\begin{align*}
    \frac{\partial f}{\partial z} & = u_x+\mi v_x = u_x-\mi u_y\\
        & = v_y+\mi v_x = v_y-\mi u_y.
\end{align*}

\section{解析函数}

\noindent \textbf{解析函数的性质}:\\
(线性性) 若 $f(z)=u+\mi v$ 在 $B$ 上解析,\\
(1) $u(x,y)=C_1$,$v(x,y)=C_2$ ($C_1$,$C_2$ 为常数) (等值线) 是 $B$ 上的两组正交曲线族。\\
(2) $u$,$v$ 均为 $B$ 上的调和函数,即 $u$,$v$ 在 $B$ 上有二阶连续偏导数,且 $\Delta u=\Delta v=0$。
$u$,$v$ 又被称为共轭调和函数。\\
\colorstar 可以从其中一个求出另一个,有三种方法:(1) 曲线积分法;(2) 凑全微分显式法;(3) 不定积分法。

\noindent \colorstar[blue] $f(z)=u(x,y)+\mi v(x,y)$ 的表达式很复杂不易转化为 $z$ 的显式表达式时,
可以令 $x=z-\mi y$ 代入化简。\\
\colorstar[blue] $\frac{1}{\rho^2}(\cos 2\theta -\mi \sin 2\theta)=\frac{1}{z^2}$\\
\colorstar[blue] $\frac{\partial x}{\partial z}=\frac{1}{2}\quad (x=\frac{z+\overline{z}}{2})$

\noindent \textbf{初等解析函数}

1. 幂函数\quad\  $w(z)=z^n$\quad $(n\in \Zs)$
\begin{equation*}
    \left\{ 
    \begin{lgathered} 
        n\geqslant 0,\quad w(z)=z^n\mbox{\ 处处解析}\\
        n < 0,\quad w(z)=z^n\mbox{\ 除\ }z=0\mbox{\ 处在复平面处处解析}
    \end{lgathered}   
    \right.
\end{equation*}
若 $s=a+\mi b\in\Cs$,$w(z)=z^s=e^{s\ln z}$。

2. 指数函数\quad\  $w(z)=e^z$
\[w=e^z=e^x(\cos y+\mi\sin y)\]
在复平面上 (\textcolor{blue}{不包括无穷远点}) 处处解析。
\begin{rmk}{}
    $e^z$ 具有虚周期 $2\mpi \mi$。
\end{rmk}

3. 三角函数
\begin{align*}
    w(z) &= \sin z =\frac{e^{\mi z}-e^{-\mi z}}{2\mi}\\
    w(z) &= \cos z =\frac{e^{\mi z}+e^{-\mi z}}{2}\\
    w(z) &= \tan z =\frac{\sin z}{\cos z}\quad (\cos z\neq 0)
\end{align*}
\begin{rmk}{}
    \ding{172} $\sin^2 z+\cos^2 z=1$,和差倍分公式、同角的不同三角函数关系、导数公式都成立;
    \ding{173} $\left\lvert \sin z\right\rvert$ 和 $\left\lvert \cos z\right\rvert$ 可能大于 $1$;
    \ding{174} $\sin z$,$\cos z$ 具有实周期 $2\mpi$。
\end{rmk}

4. 双曲函数
\begin{align*}
    \sinh z & = \frac{e^{z}-e^{-z}}{2}\\
    \cosh z & = \frac{e^{z}+e^{-z}}{2}\\
    \tanh z & = \frac{\sinh z}{\cosh z}\\
    \sech z & = \frac{1}{\cosh z}\\
    \cosech z & = \frac{1}{\sinh z}
\end{align*}
\begin{rmk}{}
    \ding{172} $\cosh^2 z - \sinh^2 z \equiv 1$;
    \ding{173} $\sinh z$,$\cosh z$ 具有纯虚数周期 $2\mpi\mi$。
\end{rmk}
\begin{equation*}
    \left\{ 
    \begin{lgathered} 
        \cos \mi x=\cosh x\\
        \sin \mi x=\mi \sinh x,
    \end{lgathered}   
    \right.\quad
    \left\{ 
    \begin{lgathered} 
        \cosh \mi x=\cos x\\
        \sinh \mi x=\mi \sin x.
    \end{lgathered}   
    \right.
\end{equation*}

5. 根式函数\quad\  $w(z)=\sqrt[n]{z},\quad n\in \Ns$
\[w=r^{\frac{1}{n}}e^{\mi\frac{\theta+2k\mpi}{n}}\quad (k=0,1,2,\dots,n-1)\]

6. 对数函数\quad\  $w(z)=\ln{z}$
\[w(z)=\ln r +\mi (\theta +2k\mpi),\quad k\in \Zs \]
\begin{rmk}{}
    多值函数,有无穷多个支,$z=0$ 和 $z=\infty$ 是两个支点。
\end{rmk}
\[\ln e^z=z+2k\mpi\mi,\quad k\in \Zs\]
\colorstar[blue] $\ln a=\ln \left\lvert a\right\rvert +\mi (2n+1)\mpi$,$a\in \Rs$ 时。

7. 一般的指数函数\quad\  $w=a^z,\quad a\neq 0,\;a\in\Cs$
\[w=e^{z\ln a}\]

8. 一般的幂函数\quad\  $w=z^s,\quad s\in\Cs$
\[w=e^{s\ln z}\]

\colorstar $\ln z$ 和 $z^{\frac{1}{n}}$ 是一对多的“函数” (多值函数)。
\section{平面标量场}

恒定场\quad 平面场\quad (\hyperref[fig:u和v]{Fig.~\ref{fig:u和v}})
\begin{figure}
    \centering
    \includegraphics{u和v.pdf}
    \caption{\label{fig:u和v} u和v}
\end{figure}
\begin{rmk}{}
    \ding{172}~$v(x,y)=C$ 是 \textasciitilde 线族,但不是该矢量的场函数。
    $v(x,y)=C$ 在这一点的方向与矢量方向相同,密集程度等于矢量的大小。
    曲线族 $u=C$ 和 $v=C$ 形成的网络叫等温网。
    \ding{173}~若已知 $h(x,y)=C$ 是其中一组等值线,那 $u$ 和 $v$ 不一定等于 $h$,
    $u$ 或 $v$ 应该为 \uwave{$F(h(x,y))$}。
\end{rmk}

\section{多值函数}

\noindent 支点:对于多值函数 $w=f(z)$,若 $z$ 绕某点一周,函数值 $w$ 不复原,
而在该点各单值分支函数值相同,则称该点为多值函数的支点。
$z$ 绕 $n$ 周复原,便称该点为 $n-1$ 阶支点。

\colorstar 支点必是奇点。

\begin{example}{}
    对于 $w=\sqrt{z}$,
    \begin{equation*}
        \left\{ 
        \begin{lgathered} 
            w_1=\sqrt{\left\lvert z\right\rvert }e^{\mi \arg \frac{z}{2}}\\
            w_2=\sqrt{\left\lvert z\right\rvert }e^{\mi \arg \frac{z}{2}+\mi \mpi},
        \end{lgathered}   
        \right.
    \end{equation*}
    $z=0$ 和 $z=\infty$ 是 $w=\sqrt{z}$ 的一阶支点。
\end{example}
\colorstar 包围了所有有限远支点,也就包围了 $\infty$ 支点。\\
$w(z)=\ln{z}$ 的支点是 $z=0,\infty$。

黎曼面 (Riemann 面) (一个两叶的面)\\
支割线 \textasciitilde 跨过此线 $z$ 辐角就增加 $2\mpi$,$w$ 的值就会改变。可以作为分割多值函数的界线。
定义一个单值的解析函数的方法。\\
\colorstar 作支割线的方法不唯一。

\begin{example}{}
    $w=\sqrt{z^2-1}+\sqrt[3]{z^2-9}$ 是\coloruline{\textcolor{red}{六}}值函数。
\end{example}
\begin{solution}{}
    几值 $2\times 3$;支点数 $2(\pm 1)\quad 3(\infty,\pm 3)$。
\end{solution}

\colorstar 判断 $\infty$ 是否为 $f(z)$ 支点时,可以作变换 $t=\frac{1}{z}$,
分析 $0$ 是否为代换后函数的支点。
\begin{example}{}
    $\infty$ 是否为 $w=\sqrt{z-a}$ 的支点。
\end{example}
\begin{solution}{}
    令 $z=\frac{1}{t}$,$w=\sqrt{\frac{1-at}{t}}$。\\
    绕 $t=0$ 转一圈后,$\arg (1-at)$ 不变,$\arg t$ 增加 $2\mpi$,
    $\arg \sqrt{t}$ 增加 $\mpi$,$w$ 减少 $\mpi$,未复原。\\
    故 $\infty$ 是 $w=\sqrt{z-a}$ 的支点。
\end{solution}

\ifx\allfiles\undefined
\end{document}
\fi