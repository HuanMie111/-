\ifx\allfiles\undefined
\documentclass[12pt, a4paper, oneside, UTF8]{ctexbook}
\def\path{../config}
\usepackage{amsmath}
\usepackage{amsthm}
\usepackage{amssymb}
\usepackage{graphicx}
\usepackage{mathrsfs}
\usepackage{mathtools}
\usepackage{pifont}  % 多种符号
\usepackage{esint}  % 各种积分号
\usepackage{upgreek}  % 正体希腊字母
\usepackage{enumitem}
\usepackage{geometry}
\usepackage[colorlinks, linkcolor=black]{hyperref}
\usepackage{stackengine}
\usepackage{yhmath}
\usepackage{extarrows}
\usepackage{unicode-math}
\usepackage{caption}
\usepackage{ulem}  % 各种标记线
\usepackage{xargs}  % 多缺省命令
\usepackage{bm}  %  加粗数学字符 

\usepackage{fancyhdr}
\usepackage[dvipsnames, svgnames]{xcolor}
\usepackage{listings}
\usepackage{zhlipsum}  % 随机中文文本, 测试用
\usepackage{lipsum}  % 随机英文文本, 测试用
\usepackage{blindtext}  % 随机科技论文,测试用
\definecolor{mygreen}{rgb}{0,0.6,0}
\definecolor{mygray}{rgb}{0.5,0.5,0.5}
\definecolor{mymauve}{rgb}{0.58,0,0.82}

\graphicspath{{figure/}, {../figure/}, {config/}, {../config/}}

% 设置图标签格式
\captionsetup[figure]{
    labelfont={bf},labelformat={default},labelsep=period,name={Fig.}}

\linespread{1.6}

\geometry{
    top=25.4mm, 
    bottom=25.4mm, 
    left=20mm, 
    right=20mm, 
    headheight=2.17cm, 
    headsep=4mm, 
    footskip=12mm
}

\setenumerate[1]{itemsep=5pt, partopsep=0pt, parsep=\parskip, topsep=5pt}
\setitemize[1]{itemsep=5pt, partopsep=0pt, parsep=\parskip, topsep=5pt}
\setdescription{itemsep=5pt, partopsep=0pt, parsep=\parskip, topsep=5pt}

\lstset{
    language=Mathematica,
    basicstyle=\tt,
    breaklines=true,
    keywordstyle=\bfseries\color{NavyBlue}, 
    emphstyle=\bfseries\color{Rhodamine},
    commentstyle=\itshape\color{black!50!white}, 
    stringstyle=\bfseries\color{PineGreen!90!black},
    columns=flexible,
    numbers=left,
    numberstyle=\footnotesize,
    frame=tb,
    breakatwhitespace=false,
} 

\numberwithin{equation}{section}  % 公式按节编号
\usepackage{tcolorbox}

\tcbuselibrary{most}

% 定义单独编号,其他四个共用一个编号计数 (原版),这里只列举了五种,其他可类似定义(未定义的使用原来的也可)
\newtcbtheorem[number within=section]{defn}
    {定义}{colback=Salmon!20, colframe=Salmon!90!Black, fonttitle=\bfseries}{def}

\newtcbtheorem[number within=section]{lemma}
    {引理}{colback=OliveGreen!10, colframe=Green!70, fonttitle=\bfseries}{lem}

% 使用另一个计数器可加参数 use counter from=lemma
\newtcbtheorem[number within=section]{them}
    {定理}{colback=SeaGreen!10!CornflowerBlue!10, 
           colframe=RoyalPurple!55!Aquamarine!100!,
           fonttitle=\bfseries}{them}

\newtcbtheorem[number within=section]{criterion}
    {准则}{colback=green!5, colframe=green!35!black, fonttitle=\bfseries}{cri}

\newtcbtheorem[number within=section]{corollary}
    {推论}{colback=Emerald!10, colframe=cyan!40!black, fonttitle=\bfseries}{cor}

\newtcbtheorem[number within=section]{proposition}
    {命题}{colback=red!5,colframe=red!75!black,fonttitle=\bfseries}{prop}
% red!5,colframe=red!75!black 警告框
% 使用格式是\begin{***}{}{} \end{***} ,需要两个 {}{} ,可以不填,但要有.
% 第一个 {} 填入别名 第二个为引用的 label 
% 引用方法为 \ref{def:xxx}

\newtheorem{example}{\indent \color{SeaGreen}{例}}[section]
\theoremstyle{plain}
\newtheorem*{rmk}{\indent 注}
\renewenvironment{proof}{\indent\textcolor{SkyBlue}{\textbf{证明:}}\;}{\qed\par}
\newenvironment{solution}{\indent\textcolor{SkyBlue}{\textbf{解:}}\;}{\qed\par}
% \def\d{\mathrm{d}}
\def\Rs{\mathbb{R}}
\def\Cs{\mathbb{C}}
\def\Zs{\mathbb{Z}}
\def\Ns{\mathbb{N}}
\newcommand\mpi{\uppi}
\newcommand\mi{\mathrm{i}} 
\newcommand\me{\mathrm{e}}
\newcommand*{\dif}{\mathop{}\!\mathrm{d}}  % 微分算符 d
% \newcommand{\bs}[1]{\boldsymbol{#1}}
\newcommand{\ora}[1]{\overrightarrow{#1}}
\newcommand{\myspace}[1]{\par\vspace{#1\baselineskip}}
\newcommand{\xrowht}[2][0]{\addstackgap[.5\dimexpr#2\relax]{\vphantom{#1}}}
\newenvironment{ca}[1][1]{\linespread{#1} \selectfont \begin{cases}}{\end{cases}}
\newenvironment{vx}[1][1]{\linespread{#1} \selectfont \begin{vmatrix}}{\end{vmatrix}}
\newcommand{\tabincell}[2]{\begin{tabular}{@{}#1@{}}#2\end{tabular}}
\newcommand{\pll}{\kern 0.56em/\kern -0.8em /\kern 0.56em}
\newcommand{\bit}[1]{\symbfit{#1}}  % 加粗斜体
\newcommand\sech{\mathrm{sech}\,}  % 正割双曲函数
\newcommand\cosech{\mathrm{cosech}\,}  % 余割双曲函数
\newcommand{\Arg}[1]{\mathrm{Arg}\;#1}  % 辐角 Arg
\newcommand{\Div}[1]{\mathrm{div}\;#1}  % 散度 div
\newcommand{\grad}[1]{\mathrm{grad}\;#1}  % 梯度 grad
\newcommand{\Rot}[1]{\mathrm{rot}\;#1}  % 旋度 rot
\newcommand{\re}[1]{\mathrm{Re}\;#1}  % 实部 Re
\newcommand{\im}[1]{\mathrm{Im}\;#1}  % 虚部 Im
\newcommand*{\colorstar}[1][black]{\textcolor{#1}{\ast\quad}}  % 星星开头的重点行 (可改变星星颜色)
% 彩色的文字及下划线
\newcommandx*{\coloruline}[3][1=black, 2=black, usedefault]{
    \textcolor{#1}{\!\uline{\textcolor{#2}{#3}}}}  % 未完成此功能,下划线换行问题

\def\myIndex{0}  % 封面
% \input{\path/cover_package_\myIndex.tex}

\def\myTitle{复变函数}
\def\myAuthor{高峰}
\def\myDateCover{2023 年 6 月 13 日}
\def\myDateForeword{\today}
\def\myForeword{前言}
\def\myForewordText{
    
}
\def\mySubheading{}


\begin{document}
% \input{\path/cover_text_\myIndex.tex}

\newpage
\thispagestyle{empty}
\begin{center}
    \Huge\textbf{\myForeword}
\end{center}
\myForewordText
\begin{flushright}
    \begin{tabular}{c}
        \myDateForeword
    \end{tabular}
\end{flushright}

\newpage
\pagestyle{plain}
\setcounter{page}{1}
\pagenumbering{Roman}
\tableofcontents

\newpage
\pagenumbering{arabic}
\setcounter{chapter}{0}  % 从第一章开始计数
\setcounter{page}{1}

\pagestyle{fancy}
\fancyfoot[C]{\thepage}
\renewcommand{\headrulewidth}{0.4pt}
\renewcommand{\footrulewidth}{0pt}








\else
\fi

\chapter{复变函数的积分}

\section{复变函数的积分}

\noindent (1) 路积分 
\[\int_{l} f(z) \dif z = \lim_{\max(\left\lvert \Delta z_k \right\rvert) \to 0} 
\sum_{k=1}^{n} f(\xi_k) \Delta z_k. \]
代入 $f(z)\to u+\mi v$,$\dif z\to \dif x +\mi \dif y$,为:
\begin{equation} \label{eq:路积分转化}
    \int_{l} f(z) \dif z = \int_{l} u \dif x - v \dif y + \mi \int_{l} v \dif x + u \dif y
\end{equation}

\noindent (2) 实变函数线积分的许多性质也对路积分成立。\\
\colorstar 路积分地位等同于实变函数的积分。\\
\colorstar 两个积分不等式
\[\left\lvert \int_{l} f(z) \dif z \right\rvert \leqslant 
\int_{l} \left\lvert f(z) \right\rvert \left\lvert \dif z \right\rvert \]
\[\left\lvert \int_{l} f(z) \dif z \right\rvert \leqslant M L \]

\section{柯西定理}

\noindent (一) 单连通区域的柯西定理

\begin{them}{单连通区域的柯西定理}{}
    如果函数 $f(z)$ 在闭单连通区域 $\overline{B}$ 上解析,则沿 $\overline{B}$ 上任一分段
    光滑闭合曲线 $l$ (也可以是 $\overline{B}$ 的边界),有 $\oint_{l} f(z) \dif z = 0$。
\end{them}
\begin{them}{单连通区域的柯西定理的一个推广}{}
    如果函数 $f(z)$ 在单连通区域 $B$ 上解析,在闭单连通区域 $\overline{B}$ 上连续,
    则沿 $\overline{B}$ 上任一分段光滑闭合曲线 $l$ (也可以是 $\overline{B}$ 的边界),
    有 $\oint_{l} f(z) \dif z = 0$。
\end{them}

\noindent (二) 复连通区域的柯西定理
\begin{them}{复连通区域的柯西定理}{}
    如果函数 $f(z)$ 是闭复连通区域上的单值解析函数,则:
    \[\oint_{l} f(z) \dif z + \sum_{i=1}^{n} \oint_{l_i} f(z) \dif z = 0, \]
    式中 $l$ 为区域外边界线,诸 $l_i$ 为区域内边界线,积分均沿边界线的正方向进行。
\end{them}
\noindent \colorstar  $\ointctrclockwise_{l} f(z) \dif z = \sum_{i=1}^{n} 
\ointctrclockwise_{l_i} f(z) \dif z$ (方向均为逆时针方向)。\\
\colorstar 对于闭单或闭复上的解析函数,若起点终点确定,且积分路径不跳过孔时,函数的积分值不变。

\section{不定积分}

\noindent $F(z) = \int_{z_0}^{z} f(\xi) \dif \xi$ (柯西型积分)\\
$F(z)$ 在其单连通区域上是解析的。\\
$F'(z)=f(z)$,$F(z)$ 是 $f(z)$ 的一个原函数\\
$\int_{z_1}^{z_2} f(\xi) \dif \xi = F(z_2) - F(z_1)$

\noindent \colorstar[blue] 两个重要式子
\begin{align*}
    \frac{1}{2\mpi\mi} \oint_{l}\;\frac{\dif z}{z-\alpha} = 
        \left\{ 
        \begin{lgathered} 0\quad (l\mbox{\ 不包围\ }\alpha) \\ 
            1\quad (l\mbox{\ 包围\ }\alpha) 
        \end{lgathered}   
        \right.\\
    \frac{1}{2\mpi\mi}\oint_{l}\;(z-\alpha)^n \dif z = 0 \quad (n\neq -1) 
\end{align*}
 
\section{柯西公式}

\begin{them}{柯西公式}{}
    若 $f(z)$ 在闭单连通区域 $\overline{B}$ 上解析,$l$ 为 $\overline{B}$ 的边界线,
    $\alpha$ 为 $\overline{B}$ 内的任一点,则有柯西公式:
    \[f(\alpha) = \frac{1}{2\mpi\mi} \oint_{l} \frac{f(z)}{z-\alpha} \dif z.\]
    也可改写为:
    \[f(z) = \frac{1}{2\mpi\mi} \oint_{l} \frac{f(\xi)}{\xi-z} \dif \xi.\]
\end{them}

\noindent 对于闭复连通区域,需将 $l$ 理解为所有边界线,且方向均为正向。\\
\colorstar 若边界 $l$ 的外部区域 (包含 $\infty$) 上解析,
有 $f(z) = \frac{1}{2\mpi\mi} \ointclockwise_{l} \frac{f(\xi)}{\xi-z} \dif \xi + f(\infty)$。\\
\colorstar 若 $z$ 不在 $l$ 内,而在 $l$ 外,则
$\oint_{l} \frac{f(\xi)}{\xi-z} \dif \xi = 0$。

\begin{corollary}{}{}
    解析函数可求导任意多次。
    \[f^{(n)}(z) = \frac{n!}{2\mpi\mi} \oint_{l} \frac{f(\xi)}{(\xi-z)^{n+1}} \dif \xi.\]
\end{corollary}

\begin{corollary}{模数定理}{}
    设 $f(z)$ 在某个闭区域上为解析,则 $\left\lvert f(z)\right\rvert $ 只能在边界线 $l$ 上取极大值。
\end{corollary}
\noindent \colorstar 只有边界 $l$ 上的 $f(z)\equiv \mbox{常数}$时,
内点的 $f(z)$ 才能与边界线的 $f(z)$ 相等。

\begin{corollary}{刘维尔定理}{}
    如 $f(z)$ 在全平面上为解析,并且是有界的,即 $\left\lvert f(z)\right\rvert \leqslant M$,
    则 $f(z)$ 必为常数。
\end{corollary}

\noindent \colorstar[red] 表示一个量或几何体随另一个量变化时,
可用下标表示。例如 P26 (2.4.2) 中的圆 $C_\varepsilon$。

\noindent \colorstar[blue] 计算路积分的方法:

(1) \textcolor{blue}{转化为线积分} (\hyperref[eq:路积分转化]{公式\,(\ref{eq:路积分转化})}) 
\textcolor{red}{($f(z)$ 不解析也可适用)}。
\begin{example}{}
    P22 例题。
\end{example}
(2) \textcolor{blue}{柯西定理 (单和复)}
\begin{example}{}
    P25 例题 $\oint_{l} \frac{1}{z-\alpha} \dif z$。
\end{example}
(3) \textcolor{blue}{柯西公式}
\begin{example}{}
    $\oint_{l} \frac{e^z}{z(z^2+1)} \dif z$,$l:\left\lvert z-\mi \right\rvert = \frac{1}{2}$。
\end{example}
\begin{solution}{}
    $\mpi (\sin 1 - \mi\cos 1)$。
\end{solution}
\begin{example}{}
    模拟题三题。
\end{example}
(4) \textcolor{blue}{$f(z)$ 的 $n$ 阶导数公式}
\begin{example}{}
    $\oint_{l} \frac{e^z}{z^n} \dif z$。
\end{example}
\begin{solution}{}
    $\frac{2\mpi \mi}{(n-1)!}$。
\end{solution}
(5) \textcolor{blue}{留数定理}

\begin{corollary}{柯西不等式}{}
    $l: \left\lvert \xi-z \right\rvert = R$,
    $f(z)$ 在 $\left\lvert \xi-z \right\rvert = R$ 内解析,
    在区域 $\left\lvert \xi-z \right\rvert \leqslant R$ 上连续,且对于 $\xi\in l$,
    $f(\xi)$ 在 $l$ 上有上界,即 $\left\lvert f(\xi) \right\rvert \leqslant M$,
    则 $\left\lvert f^{(n)}(z) \right\rvert \leqslant \frac{n! M}{R^n}$。
\end{corollary}

\begin{corollary}{中值定理(平均值定理)}{}
    若 $f(z)$ 在 $\left\lvert z-a \right\rvert < R$ 内解析,
    在 $\left\lvert z-a \right\rvert \leqslant R$ 上连续,
    则 $f(z)$ 在圆心 $a$ 处的值等于其圆周上值的平均值,即 
    $f(a) = \frac{1}{2\mpi} \int_{0}^{2\mpi} f(a + R e^{\mi \varphi}) \dif \varphi$。
\end{corollary}

\begin{corollary}{Morera 定理 \textcolor{red}{(柯西定理的逆定理)}}{}
    设 $f(z)$ 在 $\sigma$ 上连续,对 $\sigma$ 内任意闭合轨道 $l$ 的路积分为 $0$,
    即 $\oint_{l} f(z) \dif z = 0$,则 $f(z)$ 在 $\sigma$ 内解析。
\end{corollary}
\noindent \colorstar[red] 可以用来证明 $\sigma$ 解析。\\
\colorstar[blue] 对于 $\oint_{\left\lvert z \right\rvert = 3} \frac{1}{z^3(z^2-1)} \dif z$,
先使用复柯定理将大路积分转化为三个小路积分,再各自使用柯西公式去求。\\
\colorstar[blue] 注意讨论 $l$ 是否包围奇点。

\ifx\allfiles\undefined
\end{document}
\fi