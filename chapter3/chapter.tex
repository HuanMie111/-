\ifx\allfiles\undefined
\documentclass[12pt, a4paper, oneside, UTF8]{ctexbook}
\def\path{../config}
\usepackage{amsmath}
\usepackage{amsthm}
\usepackage{amssymb}
\usepackage{graphicx}
\usepackage{mathrsfs}
\usepackage{mathtools}
\usepackage{pifont}  % 多种符号
\usepackage{esint}  % 各种积分号
\usepackage{upgreek}  % 正体希腊字母
\usepackage{enumitem}
\usepackage{geometry}
\usepackage[colorlinks, linkcolor=black]{hyperref}
\usepackage{stackengine}
\usepackage{yhmath}
\usepackage{extarrows}
\usepackage{unicode-math}
\usepackage{caption}
\usepackage{ulem}  % 各种标记线
\usepackage{xargs}  % 多缺省命令
\usepackage{bm}  %  加粗数学字符 

\usepackage{fancyhdr}
\usepackage[dvipsnames, svgnames]{xcolor}
\usepackage{listings}
\usepackage{zhlipsum}  % 随机中文文本, 测试用
\usepackage{lipsum}  % 随机英文文本, 测试用
\usepackage{blindtext}  % 随机科技论文,测试用
\definecolor{mygreen}{rgb}{0,0.6,0}
\definecolor{mygray}{rgb}{0.5,0.5,0.5}
\definecolor{mymauve}{rgb}{0.58,0,0.82}

\graphicspath{{figure/}, {../figure/}, {config/}, {../config/}}

% 设置图标签格式
\captionsetup[figure]{
    labelfont={bf},labelformat={default},labelsep=period,name={Fig.}}

\linespread{1.6}

\geometry{
    top=25.4mm, 
    bottom=25.4mm, 
    left=20mm, 
    right=20mm, 
    headheight=2.17cm, 
    headsep=4mm, 
    footskip=12mm
}

\setenumerate[1]{itemsep=5pt, partopsep=0pt, parsep=\parskip, topsep=5pt}
\setitemize[1]{itemsep=5pt, partopsep=0pt, parsep=\parskip, topsep=5pt}
\setdescription{itemsep=5pt, partopsep=0pt, parsep=\parskip, topsep=5pt}

\lstset{
    language=Mathematica,
    basicstyle=\tt,
    breaklines=true,
    keywordstyle=\bfseries\color{NavyBlue}, 
    emphstyle=\bfseries\color{Rhodamine},
    commentstyle=\itshape\color{black!50!white}, 
    stringstyle=\bfseries\color{PineGreen!90!black},
    columns=flexible,
    numbers=left,
    numberstyle=\footnotesize,
    frame=tb,
    breakatwhitespace=false,
} 

\numberwithin{equation}{section}  % 公式按节编号
\usepackage{tcolorbox}

\tcbuselibrary{most}

% 定义单独编号,其他四个共用一个编号计数 (原版),这里只列举了五种,其他可类似定义(未定义的使用原来的也可)
\newtcbtheorem[number within=section]{defn}
    {定义}{colback=Salmon!20, colframe=Salmon!90!Black, fonttitle=\bfseries}{def}

\newtcbtheorem[number within=section]{lemma}
    {引理}{colback=OliveGreen!10, colframe=Green!70, fonttitle=\bfseries}{lem}

% 使用另一个计数器可加参数 use counter from=lemma
\newtcbtheorem[number within=section]{them}
    {定理}{colback=SeaGreen!10!CornflowerBlue!10, 
           colframe=RoyalPurple!55!Aquamarine!100!,
           fonttitle=\bfseries}{them}

\newtcbtheorem[number within=section]{criterion}
    {准则}{colback=green!5, colframe=green!35!black, fonttitle=\bfseries}{cri}

\newtcbtheorem[number within=section]{corollary}
    {推论}{colback=Emerald!10, colframe=cyan!40!black, fonttitle=\bfseries}{cor}

\newtcbtheorem[number within=section]{proposition}
    {命题}{colback=red!5,colframe=red!75!black,fonttitle=\bfseries}{prop}
% red!5,colframe=red!75!black 警告框
% 使用格式是\begin{***}{}{} \end{***} ,需要两个 {}{} ,可以不填,但要有.
% 第一个 {} 填入别名 第二个为引用的 label 
% 引用方法为 \ref{def:xxx}

\newtheorem{example}{\indent \color{SeaGreen}{例}}[section]
\theoremstyle{plain}
\newtheorem*{rmk}{\indent 注}
\renewenvironment{proof}{\indent\textcolor{SkyBlue}{\textbf{证明:}}\;}{\qed\par}
\newenvironment{solution}{\indent\textcolor{SkyBlue}{\textbf{解:}}\;}{\qed\par}
% \def\d{\mathrm{d}}
\def\Rs{\mathbb{R}}
\def\Cs{\mathbb{C}}
\def\Zs{\mathbb{Z}}
\def\Ns{\mathbb{N}}
\newcommand\mpi{\uppi}
\newcommand\mi{\mathrm{i}} 
\newcommand\me{\mathrm{e}}
\newcommand*{\dif}{\mathop{}\!\mathrm{d}}  % 微分算符 d
% \newcommand{\bs}[1]{\boldsymbol{#1}}
\newcommand{\ora}[1]{\overrightarrow{#1}}
\newcommand{\myspace}[1]{\par\vspace{#1\baselineskip}}
\newcommand{\xrowht}[2][0]{\addstackgap[.5\dimexpr#2\relax]{\vphantom{#1}}}
\newenvironment{ca}[1][1]{\linespread{#1} \selectfont \begin{cases}}{\end{cases}}
\newenvironment{vx}[1][1]{\linespread{#1} \selectfont \begin{vmatrix}}{\end{vmatrix}}
\newcommand{\tabincell}[2]{\begin{tabular}{@{}#1@{}}#2\end{tabular}}
\newcommand{\pll}{\kern 0.56em/\kern -0.8em /\kern 0.56em}
\newcommand{\bit}[1]{\symbfit{#1}}  % 加粗斜体
\newcommand\sech{\mathrm{sech}\,}  % 正割双曲函数
\newcommand\cosech{\mathrm{cosech}\,}  % 余割双曲函数
\newcommand{\Arg}[1]{\mathrm{Arg}\;#1}  % 辐角 Arg
\newcommand{\Div}[1]{\mathrm{div}\;#1}  % 散度 div
\newcommand{\grad}[1]{\mathrm{grad}\;#1}  % 梯度 grad
\newcommand{\Rot}[1]{\mathrm{rot}\;#1}  % 旋度 rot
\newcommand{\re}[1]{\mathrm{Re}\;#1}  % 实部 Re
\newcommand{\im}[1]{\mathrm{Im}\;#1}  % 虚部 Im
\newcommand*{\colorstar}[1][black]{\textcolor{#1}{\ast\quad}}  % 星星开头的重点行 (可改变星星颜色)
% 彩色的文字及下划线
\newcommandx*{\coloruline}[3][1=black, 2=black, usedefault]{
    \textcolor{#1}{\!\uline{\textcolor{#2}{#3}}}}  % 未完成此功能,下划线换行问题

\def\myIndex{0}  % 封面
% \input{\path/cover_package_\myIndex.tex}

\def\myTitle{复变函数}
\def\myAuthor{高峰}
\def\myDateCover{2023 年 6 月 13 日}
\def\myDateForeword{\today}
\def\myForeword{前言}
\def\myForewordText{
    
}
\def\mySubheading{}


\begin{document}
% \input{\path/cover_text_\myIndex.tex}

\newpage
\thispagestyle{empty}
\begin{center}
    \Huge\textbf{\myForeword}
\end{center}
\myForewordText
\begin{flushright}
    \begin{tabular}{c}
        \myDateForeword
    \end{tabular}
\end{flushright}

\newpage
\pagestyle{plain}
\setcounter{page}{1}
\pagenumbering{Roman}
\tableofcontents

\newpage
\pagenumbering{arabic}
\setcounter{chapter}{0}  % 从第一章开始计数
\setcounter{page}{1}

\pagestyle{fancy}
\fancyfoot[C]{\thepage}
\renewcommand{\headrulewidth}{0.4pt}
\renewcommand{\footrulewidth}{0pt}








\else
\fi

\chapter{幂级数展开}

\section{复数项级数}

\noindent 复数项无穷级数:
\[\sum_{k=0}^{\infty} w_k = w_0 + w_1 + w_2 + \dots + w_k + \cdots, \quad (1)\] 
其中每一项 $w_k=u_k+\mi v_k$。\\
部分和 $W_n = \sum_{k=0}^\infty w_k$\\
\colorstar 复数项无穷级数的收敛性问题可以归结为2个实数项级数的收敛性问题。\\
\colorstar $\lim_{k\to\infty}\left\lvert w_k\right\rvert = 0$ 是 
级数\,(1)\,收敛的必要条件。

\begin{criterion}{柯西收敛判据}{}
    级数\,(1)\,收敛的充要条件是:对于 $\forall$ 小正数 $\varepsilon$,
    必有 $N$ 存在,使得 $n>N$ 时,$\left\lvert \sum_{k=n+1}^{n+p} w_k\right\rvert < \varepsilon$,
    其中 $p$ 为任意正整数。
\end{criterion}

\noindent \textbf{绝对收敛}\\
若 $\left\lvert \sum_{k=0}^{\infty} w_k\right\rvert$ 收敛,则级数\,(1)\,绝对收敛。\\
\colorstar 绝对收敛的级数必收敛。\\
\colorstar 绝对收敛的级数各项先后次序可以任意改变,其和不变。\\
\colorstar 两个绝对收敛的级数,$\sum_{k=0}^{\infty} p_k = A$ 和 $\sum_{l=0}^{\infty} q_l = B$,
逐项相乘得到的新级数也绝对收敛,且 $\sum_{k=0}^{\infty} p_k \cdot \sum_{l=0}^{\infty} q_l 
= \sum_{k=0}^{\infty}\sum_{l=0}^{\infty} p_k q_l = \sum_{n=0}^{\infty} c_n = A B$,
其中 $c_n = \sum_{k=0}^{n} p_k q_{n-k}$。

\begin{criterion}{比值判别法 (达朗贝尔判别法)}{}
    \begin{equation*} 
        \lim_{k\to \infty} \left\lvert \frac{w_{k+1}}{w_k} \right\rvert = l
            \left\{ 
            \begin{lgathered} 
                < 1\quad \mbox{则\ } \sum_{k=0}^{\infty} w_k\mbox{\ 绝对收敛},\\ 
                > 1\quad \mbox{则\ } \sum_{k=0}^{\infty} w_k\mbox{\ 发散}. 
            \end{lgathered}   
            \right.
    \end{equation*}
\end{criterion}

\begin{criterion}{根值判别法 (柯西判别法)}{}
    \begin{equation*} 
        \lim_{k\to \infty} \sqrt[k]{\left\lvert w_k \right\rvert} = r 
        \left\{ 
            \begin{lgathered} 
                < 1\quad\mbox{则绝对收敛},\\ 
                > 1\quad\mbox{则发散}. 
            \end{lgathered}   
        \right.
    \end{equation*}
\end{criterion}

\noindent \colorstar 比值判别法 $l=1$ 的情况下,可以使用高斯判别法判断。

\begin{criterion}{高斯判别法}{}
    \begin{equation*} 
        \frac{w_k}{w_{k+1}} = 1 + \frac{\mu}{k} + \mathcal{o}(\frac{1}{k}), \mbox{\ 若}
            \left\{ 
            \begin{lgathered} 
                \re \mu > 1\quad \mbox{则绝对收敛},\\ 
                \re \mu \leqslant 1\quad \mbox{则发散}. 
            \end{lgathered}   
            \right.
    \end{equation*}
\end{criterion}

\noindent 函数项级数:
\[\sum_{k=0}^{\infty} w_k (z) = w_0 (z) + w_1 (z) + w_2 (z) + \dots + w_k (z) + \cdots. \quad (2)\]
在某个区域 $B$ (或 $l$) 上所有点\,(2)\,收敛,则称\,(2)\,在 $B$ (或 $l$) 上收敛。

\noindent \textbf{收敛与一致收敛}\\
已知\,(2)\,一致收敛,\\
性质 (1):若每一项 $w_k(z)$ 在 $l$ 上连续,则级数的和 $w(z)$ 也在 $l$ 上连续,
且 $\int_l w(z) \dif z = \sum_{k=0}^{\infty} \int_l w_k (z) \dif z$。
\textcolor{blue}{(逐项积分)}\\
性质 (2):若 $w_k(z)$ 在 $B$ 上连续,则 $w(z)$ 也在 $B$ 上连续。\\
性质 (3):若\,(2)\,在 $\overline{B}$ 上一致收敛,$w_k(z)$ 在 $\overline{B}$ 中单值解析,
则 $w(z)$ 也是 $\overline{B}$ 上的单值解析函数,
且 $w^{(n)}(z) = \sum_{k=0}^{\infty} w^{(n)}_k (z)$,
且 $w^{(n)}(z)$ 在 $\overline{B}$ 内的任何一个闭区域中一致收敛。
\textcolor{blue}{(逐项求导)}

\noindent \textbf{绝对且一致收敛}

\begin{criterion}{m 判别法 \textcolor{blue}{(充分但不必要)}}{}
    如果对于某个区域 $B$ (或 $l$) 上所有点 $z$,(2)\,的各项的模 
    $\left\lvert w_k (z) \right\rvert \leqslant m_k$,
    且 $\sum_{k=0}^{\infty} m_k$ 收敛,则\,(2)\,在 $B$ (或 $l$) 上绝对且一致收敛。
\end{criterion}

\noindent \colorstar[red] \textcolor{red}{一致收敛的证明方法 (1)}

\section{幂级数} 
\noindent 幂级数 (以 $z_0$ 为中心的幂级数):
\[\sum_{k=0}^{\infty} a_k (z-z_0)^k = a_0 + a_1 (z-z_0) + a_2 (z-z_0)^2 + \cdots, \quad (3)\]
其中 $a_0$, $z_0$, $a_1$, $\cdots$ 都是复常数。\\
收敛半径 $R = \lim_{k\to \infty} \left\lvert \frac{a_k}{a_{k+1}} \right\rvert 
= \lim_{k\to \infty} \frac{1}{\sqrt[k]{\left\lvert a_k \right\rvert}}$ 

\begin{them}{阿贝尔 (Abel) 定理}{}
    若\,(3)\,在圆周上一点收敛,则在圆内绝对收敛,在更小的闭圆上一致收敛。
    (如 \hyperref[fig:幂级数的收敛域]{Fig.~\ref{fig:幂级数的收敛域}} 所示)
\end{them}

\begin{figure}
    \centering
    \includegraphics{幂级数的收敛域.pdf}
    \caption{\label{fig:幂级数的收敛域} 幂级数的收敛域}
\end{figure}
\noindent \colorstar[red] \textcolor{red}{一致收敛的证明方法 (2)}

\noindent \colorstar $\sum_{k=0}^{\infty} t^k = 1 + t + t^2 + \dots + t^k + \dots = \frac{1}{1-t} 
\quad (\left\lvert t \right\rvert < 1)$\\
\colorstar p级数 $\sum_{k=1}^{\infty} \frac{1}{k^p}$,$p\in \Rs$,
当 $p\geqslant 2$ 时,收敛。

\[\frac{1}{2\mpi\mi} \oint_{C_{R_1}}\frac{w(\xi)}{\xi-z}\dif \xi 
= a_0 + a_1 (z-z_0) + \dots = w(z)\]
($C_{R_1} $为半径小于 $1$ 的圆)\\
幂级数的和函数在收敛圆内部解析。\\
幂级数逐项求导或逐项积分不改变 $R$。

\section{泰勒级数展开}

\begin{them}{泰勒展开定理}{}
    设 $f(z)$ 在以 $z_0$ 为圆心的圆 $C_R$ 内解析,则对圆内的任意 $z$ 点,$f(z)$ 可展为幂级数:
\[f(z) = \sum_{k=0}^{\infty} a_k (z-z_0)^k,\]
    其中,$a_k = \frac{1}{2\mpi\mi} \oint_{C_{R_1}}\;\frac{f(\xi)}{(\xi-z_0)^{k+1}}\dif \xi 
    = \frac{f^{(k)} z_0}{k!}$,$C_{R_1}$ 为 $C_R$ 内包含 $z$ 且与 $C_R$ 同心的圆。
\end{them}
\noindent \colorstar 泰勒展开是唯一的。\\
\textbf{泰勒展开} $f(z) = \sum_{k=0}^{\infty} \frac{f^{(k)} (z_0)}{k!} 
(z-z_0)^k \quad (\left\lvert z-z_0 \right\rvert < R)$,
以 $z_0$ 为中心的 \textcolor{blue}{泰勒级数}。

\noindent 一些初等函数的泰勒展开:
\begin{align*}
    e^z & = \sum_{k=0}^{\infty} \frac{z_k}{k!}\\
    \sin z & = \sum_{k=0}^{\infty} \frac{(-1)^k z^{2k+1}}{(2k+1)!}\\
    \cos z & = \sum_{k=0}^{\infty} \frac{(-1)^k z^{2k}}{(2k)!}\\
    \frac{1}{1-z} & = \sum_{k=0}^{\infty} z^k \quad \textcolor{blue}{(\left\lvert z \right\rvert < 1)}\\
    \frac{1}{1+z} & = \sum_{k=0}^{\infty} (-1)^k z^k \quad \textcolor{blue}{(\left\lvert z \right\rvert < 1)}\\
    \ln z & = 2n\mpi\mi + \sum_{k=1}^{\infty} \frac{(-1)^{k+1} (z-1)^k}{k} \quad \textcolor{blue}{(\left\lvert z-1 \right\rvert < 1)}\\
    (1+z)^m & = 1^m (\sum_{k=0}^{\infty} \frac{m! z^k}{(m-k)!k!}), \quad \textcolor{blue}{(\left\lvert z \right\rvert < 1)}
\end{align*}
其中 $1^m=e^{\mi mn 2\mpi}$ ($n$ 为整数),\textcolor{blue}{$m$ 不是整数}。
(\textcolor{red}{指数为非整数的二项式定理})

\noindent \colorstar[blue] 展开为泰勒级数/洛朗级数的方法:\\
(1) \textcolor{blue}{定义。求各阶导数代入。}\\
(2) \textcolor{blue}{利用初等函数的泰勒展开、变形和代换。}\\
(3) \textcolor{blue}{逐项求导和积分。}

\noindent \colorstar 柯西乘积表\\
$(\sum_{k=0}^{\infty} a_k z^k)\cdot (\sum_{k=0}^{\infty} b_k z^k) = (\sum_{k=0}^{\infty} c_k z^k)$,
其中 $c_k = \sum_{n=0}^{k} a_n b_{k-n}$ (\hyperref[fig:柯西乘积表的系数]{Fig.~\ref{fig:柯西乘积表的系数}})。

\begin{figure}
    \centering
    \includegraphics{柯西乘积表的系数ck.pdf}
    \caption{\label{fig:柯西乘积表的系数} 柯西乘积表的系数 $c_k$}
\end{figure}

\noindent \colorstar[blue] 求幂级数和函数的方法:\\
(1) \textcolor{blue}{初等函数的泰勒展开的代换和变形}\\
(2) \textcolor{blue}{柯西乘积表}
\begin{example}{}
    $\frac{1}{1-z^2} = (\frac{1}{1-z})(\frac{1}{1+z})$
\end{example}
\noindent (3) \textcolor{blue}{逐项求导或积分。}

\section{解析延拓}

\begin{defn}{解析延拓}{}
    若 $f_1(z)$ 在 $\sigma_1$ 解析,$f_2(z)$ 在 $\sigma_2$ 解析,
    且 $\sigma_1 \cap \sigma_2=\sigma_3\neq \phi$,在 $\sigma_3$ 上有$f_1(z)\equiv f_2(z)$,
    则 $f_1(z)$ 和 $f_2(z)$ 互为彼此的解析延拓。
\end{defn}

解析延拓是唯一的。

\noindent \colorstar[blue] 解析延拓的方法:\\
\ding{172} 泰勒级数展开\\
\ding{173} 用函数关系:例如 $\Gamma(z)$\\
\ding{173} 许瓦兹反射原理:若 $f(z)$ 在包括实轴在内的上半平面解析,
且 $R'_2 \leqslant \left\lvert z-z_0 \right\rvert \leqslant R'_1$ 内一致收敛, 
$f(z)$ 在实轴上值是实数,则 $\overline{f(\overline{z})}$ 是下半平面的解析延拓。

\section{洛朗级数}

\noindent 双边幂级数:
\[\cdots + a_{-2} (z-z_0)^{-2} + a_{-1} (z-z_0)^{-1} + a_0 + a_{1} (z-z_0) + a_{2} (z-z_0)^{2} + \cdots. \quad (4)\]
\textcolor{blue}{无限部分/主部:$\cdots + a_{-2} (z-z_0)^{-2} + a_{-1} (z-z_0)^{-1}$。
$> R_2$,在更大的开域圆外一致收敛;}\\
\textcolor{blue}{解析部分/正则部:$a_0 + a_{1} (z-z_0) + a_{2} (z-z_0)^{2} + \cdots$。
$< R_1$,在更小的闭域圆内一致收敛。}

\begin{align*}
    \left\{ 
    \begin{lgathered} 
        \mbox{若\ }R_2<R_1,\quad R_2<\left\lvert z-z_0 \right\rvert <R_1 \mbox{\ 收敛环\ }\\ 
        \mbox{若\ }R_2>R_1,\quad \mbox{则级数\,(4) 处处发散}
    \end{lgathered}   
    \right.
\end{align*}

\begin{defn}{幂级数展开定理}{}
    设 $f(z)$ 在环形区域 $R_2<\left\lvert z-z_0 \right\rvert <R_1$ 的内部单值解析,
    则对环域上任一点 $z$, $f(z)$ 可展为幂级数:
    \[\mbox{\textcolor{blue}{洛朗展开}\quad}
        f(z) = \sum_{k=-\infty}^{\infty} a_k (z-z_0)^k,
    \mbox{\quad \textcolor{blue}{洛朗级数}}\]
    其中 $a_k = \frac{1}{2\mpi\mi} \oint_{C}\;\frac{f(\xi)}{(\xi-z_0)^{k+1}}\dif \xi$, 
    $C$ 为开环域内逆时针方向绕内圆一周的任一闭合曲线。    
\end{defn}
\begin{rmk}{}
    \ding{172} 洛朗展开是唯一的;\ding{173} 两种展开的 $a_k$ 值不一样。
\end{rmk}

\noindent \colorstar[blue] 求洛朗展开的步骤 
(\hyperref[fig:求洛朗级数的步骤]{Fig.~\ref{fig:求洛朗级数的步骤}})\\
\ding{172} 找出所有奇点\\
\ding{173} 对所有环域分别进行洛朗展开。(中心点若为解析点,中心域上则是一个泰勒展开)
\begin{figure}
    \centering
    \includegraphics{求洛朗级数的步骤.pdf}
    \caption{\label{fig:求洛朗级数的步骤} 求洛朗级数的步骤}
\end{figure}

\section{孤立奇点的分类}

孤立奇点: $f(z)$ 在 $z_0$ 不可导,而在 $z_0$ 的任意小邻域内除 $z_0$ 外处处可导。

非孤立奇点:在 $z_0$ 无论多么小的邻域内总可以找到除 $z_0$ 以外的不可导点。
\begin{example}{}
    $f(z) = \frac{1}{\sin \frac{1}{z}}$,$z_0=0$,
    $z_k=\frac{1}{k\mpi}$ (\textcolor{blue}{非孤立奇点}),$k=1,2,\cdots$.
\end{example}

\begin{align*}
    \mbox{孤立奇点}
        \left\{ 
        \begin{lgathered} 
            \mbox{可去奇点:没有负幂项}\\ 
            \mbox{极点:有限个负幂项}\\
            \mbox{本性奇点:无限个负幂项}
        \end{lgathered}   
        \right.
\end{align*}
\begin{rmk}{}
    \ding{172} $z_0$ 为可去奇点,\uwave{$\lim_{z\to z_0} f(z) = a_0$}。
    函数在可去奇点的邻域上是有界的。可去奇点可以不作为奇点看待。\\
    \ding{173} $z_0$ 为极点时,\uwave{$\lim_{z\to z_0} f(z) = \infty$}
    \textcolor{blue}{(可作为奇点类型的判据)}。 阶,单极点\\
    \colorstar 若主部可表示为 $\frac{\varphi(z)}{(z-z_0)^m}$,$\varphi(z_0)\neq 0$,则为 $m$ 阶。\\
    若主部可表示为 $\frac{\varphi(z)}{\psi (z)}$,$\varphi(z_0)\neq 0$,$\psi (z_0) = 0$,
    且 $z_0$ 是 $\psi (z)$ 的 $m$ 重零根,则为 $m$ 阶。\\
    若 $z_0$ 是 $\frac{1}{f(z)}$ 的 $m$ 重零点,则 $z_0$ 是 $f(z)$ 的 $m$ 阶极点。\\
    \ding{174} $z_0$ 为本性奇点,\uwave{$\lim_{z\to z_0} f(z)$\ 不存在}。\\
    \ding{175} 上述 $z_0$ 都是有限远点。
\end{rmk}

$\infty$ 展开的洛朗级数正则部和主部名称互换,对应也有各种奇点。\\
\ding{172} $\lim_{z\to \infty} f(z)$ 存在且有限,$\infty$ 为可去奇点。\\
\ding{173} $f(z)$ 最高正幂次方为 $m$,$\infty$ 为 $m$ 阶极点。\\
\ding{174} 若 $m$ 为 $\infty$,则 $\infty$ 为本性奇点。

支点:解析型;极点型;本性奇点型

在支点邻域的洛朗级数展开式,幂指数为分数。

\noindent \textbf{$\Gamma$ 函数} --- 第二类欧拉型积分
\[\Gamma(z) = \int_0^\infty t^{z-1}e^{-t} \dif t,\quad \re z > 0\]
基本性质:\\
\ding{172} $\Gamma(1) = 1$\\
\ding{173} $\Gamma(z+1) = z\cdot \Gamma(z)$\\
\ding{174} $\Gamma(n) = (n-1)!,\quad n\in \Ns$\\
\ding{175} $\Gamma(z)\Gamma(1-z) = \frac{\mpi}{\sin (\mpi z)}$\\
\ding{176} $\Gamma(\frac{1}{2}) = \sqrt{\mpi} = 2\int_0^\infty e^{-r^2} \dif r 
= \int_{-\infty}^\infty e^{-r^2} \dif r$\\
$\Gamma(\frac{2n+1}{2}) = \frac{(2n-1)!!}{2^n} \sqrt{\mpi}$
$\int_0^\infty r^p e^{-r^2} \dif r = \frac{1}{2} \Gamma(\frac{p+1}{2})$

\noindent \textbf{Beta 函数}
\[B(p,q) = \int_0^1 t^{p-1} (1-t)^{q-1} \dif t. \quad (\re p > 0, \re q > 0)\]
$B(p,q) = \frac{\Gamma(p)\Gamma(q)}{\Gamma(p+q)}$

\ifx\allfiles\undefined
\end{document}
\fi